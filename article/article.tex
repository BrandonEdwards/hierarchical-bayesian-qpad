% Options for packages loaded elsewhere
\PassOptionsToPackage{unicode}{hyperref}
\PassOptionsToPackage{hyphens}{url}
%
\documentclass[
  11pt,
]{article}
\usepackage{lmodern}
\usepackage{amssymb,amsmath}
\usepackage{ifxetex,ifluatex}
\ifnum 0\ifxetex 1\fi\ifluatex 1\fi=0 % if pdftex
  \usepackage[T1]{fontenc}
  \usepackage[utf8]{inputenc}
  \usepackage{textcomp} % provide euro and other symbols
\else % if luatex or xetex
  \usepackage{unicode-math}
  \defaultfontfeatures{Scale=MatchLowercase}
  \defaultfontfeatures[\rmfamily]{Ligatures=TeX,Scale=1}
\fi
% Use upquote if available, for straight quotes in verbatim environments
\IfFileExists{upquote.sty}{\usepackage{upquote}}{}
\IfFileExists{microtype.sty}{% use microtype if available
  \usepackage[]{microtype}
  \UseMicrotypeSet[protrusion]{basicmath} % disable protrusion for tt fonts
}{}
\makeatletter
\@ifundefined{KOMAClassName}{% if non-KOMA class
  \IfFileExists{parskip.sty}{%
    \usepackage{parskip}
  }{% else
    \setlength{\parindent}{0pt}
    \setlength{\parskip}{6pt plus 2pt minus 1pt}}
}{% if KOMA class
  \KOMAoptions{parskip=half}}
\makeatother
\usepackage{xcolor}
\IfFileExists{xurl.sty}{\usepackage{xurl}}{} % add URL line breaks if available
\IfFileExists{bookmark.sty}{\usepackage{bookmark}}{\usepackage{hyperref}}
\hypersetup{
  pdftitle={Hierarchical QPAD},
  pdfauthor={Brandon P.M. Edwards*1,2; 1Department of Biology, Carleton University, Ottawa, ON, Canada; 2Canadian Wildlife Service, Environment and Climate Change Canada, Ottawa, ON, Canada},
  hidelinks,
  pdfcreator={LaTeX via pandoc}}
\urlstyle{same} % disable monospaced font for URLs
\usepackage[margin=1in]{geometry}
\usepackage{graphicx,grffile}
\makeatletter
\def\maxwidth{\ifdim\Gin@nat@width>\linewidth\linewidth\else\Gin@nat@width\fi}
\def\maxheight{\ifdim\Gin@nat@height>\textheight\textheight\else\Gin@nat@height\fi}
\makeatother
% Scale images if necessary, so that they will not overflow the page
% margins by default, and it is still possible to overwrite the defaults
% using explicit options in \includegraphics[width, height, ...]{}
\setkeys{Gin}{width=\maxwidth,height=\maxheight,keepaspectratio}
% Set default figure placement to htbp
\makeatletter
\def\fps@figure{htbp}
\makeatother
\setlength{\emergencystretch}{3em} % prevent overfull lines
\providecommand{\tightlist}{%
  \setlength{\itemsep}{0pt}\setlength{\parskip}{0pt}}
\setcounter{secnumdepth}{-\maxdimen} % remove section numbering
% load packages
\usepackage{amsmath,amsfonts,float,makecell,titletoc,titlesec,tocloft,lineno,booktabs,subfiles,textcomp,tabularx,etoolbox,longtable, adjustbox}
\usepackage[T1]{fontenc}
\usepackage{lmodern}
\usepackage[utf8]{inputenc}
\usepackage[singlespacing]{setspace}
\usepackage[autostyle]{csquotes}

% format captions
\usepackage[labelfont={small,bf}, labelsep=space, font={small}]{caption}

% allow breaks in equations
\allowdisplaybreaks

% format section headers
\titleformat*{\section}{\large\bfseries}
\titleformat*{\subsection}{\large\bfseries}

% make figures static
\let\origfigure\figure
\let\endorigfigure\endfigure
\renewenvironment{figure}[1][2] {
\expandafter\origfigure\expandafter[H]
} {
\endorigfigure
}

% center tables
\let\Begin\begin

\let\End\end

% make tables in fontnote size font
\AtBeginEnvironment{longtable}{\scriptsize\singlespacing}
\AtBeginEnvironment{multicols}{\scriptsize\singlespacing}
\AtBeginEnvironment{tabular}{\scriptsize\singlespacing}
\AtBeginEnvironment{longtabu}{\scriptsize\singlespacing}

% define struts for tables
\newcommand\T{\rule{0pt}{2.6ex}} % top strut
\newcommand\B{\rule[-1.2ex]{0pt}{0pt}} % bottom strut

% add extra space between rows
\renewcommand{\arraystretch}{1.2}

% fix author spacing
\makeatletter
\def\and{%
  \end{tabular}%
  \vskip -2.5em \hskip 0.1em \@plus.17fil\relax
  \begin{tabular}[t]{c}}
\makeatother

% format urls
\urlstyle{same}
\hypersetup{urlcolor=black} % Does not apply color to href's
\makeatletter
\g@addto@macro{\UrlBreaks}{\UrlOrds\do\.\do\@\do\\\do\/\do\!\do\_\do\|\do\;\do\>\do\]\do\)\do\,\do\?\do\'\do+\do\=\do\#}
% \do-\do\/\do\\\do\.\do=}
\makeatother

% table seperators
\newcommand{\oldtabcolsep}{\tabcolsep}

\title{Hierarchical QPAD}
\author{\normalsize Brandon P.M. Edwards*\textsuperscript{1,2} \and \small \textsuperscript{1}Department of Biology, Carleton University,
Ottawa, ON, Canada \and \small \textsuperscript{2}Canadian Wildlife Service, Environment and
Climate Change Canada, Ottawa, ON, Canada}
\date{26 October 2021}

\begin{document}
\maketitle

\begin{center}\rule{0.5\linewidth}{0.5pt}\end{center}

\doublespacing
\vfill

\hypertarget{manuscript-information}{%
\section{MANUSCRIPT INFORMATION}\label{manuscript-information}}

Format: Methods in Ecology and Evolution (research article) \newline
Running headline: Bayesian QPAD \newline Abstract word count: 0 / 350
\newline Main text word count: 0 / 7,000 (including captions and
references) \newline Number of references: 0 \newline Number of figures:
0 / 6 \newline Keywords: TO DO \newline Spelling: American English
(en-US)

\clearpage
\linenumbers

\hypertarget{abstract}{%
\section{ABSTRACT}\label{abstract}}

\begin{spacing}{1.5}

TO DO

\end{spacing}

\clearpage

\hypertarget{introduction}{%
\section{INTRODUCTION}\label{introduction}}

TO DO

\clearpage

\hypertarget{methods}{%
\section{METHODS}\label{methods}}

\hypertarget{single-species-modelling}{%
\subsection{Single Species Modelling}\label{single-species-modelling}}

The single species Bayesian QPAD modelling is developed similarly to the
conditional maximum likelihood approach in (Sólymos \emph{et al.} 2013).
For posterity, we will derive the model here, including choice of priors
for relevant parameters.

Let \(Y_{ijk}\) be the count of a given species during sampling event
\(i\), occuring in time band \(j \in [1,J]\) and/or distance band
\(k \in [1,K]\).

\hypertarget{removal-modelling}{%
\subsubsection{Removal Modelling}\label{removal-modelling}}

For removal modelling, we sum counts over all distance bands; that is,
we have \(Y_{ij.} = \sum_K Y_{ijk}\). Let \(\pi_{ij}\) be the
probability that an individual \(y\) is a member of the set of
individuals in \(Y_{ij.}\), given it is a member of the set of
individuals in the total count \(Y_{i..}\). Then, the density function
for the removal model is given by

\[
  Y_{ij.}\sim multinomial(Y_{i..}, \vec{\pi}_{i})
\]

Similar to (Sólymos \emph{et al.} 2013), let \(t_{ij}\) be the maximum
time for time band \(j\) during sampling event \(i\), and let \(\phi_i\)
be the unknown cue rate. Then, we have

\[
  \pi_{ij} = \begin{cases}
  \dfrac{\exp\{-t_{ij-1} \phi_i\} - \exp\{-t_{ij}\phi_i\}}{1-\exp\{-t_{Ji}\phi_i\}}, &\text{$j>1$} \\
  1-\sum_{a=2}^J \pi_a, &\text{$j=1$}
  \end{cases}
\]

TO DO

\hypertarget{qpad-distance-modelling}{%
\subsection{QPAD Distance Modelling}\label{qpad-distance-modelling}}

TO DO

\hypertarget{model-validation}{%
\subsection{Model Validation}\label{model-validation}}

We developed three designed experiments to both compare a Bayesian
implementation of the QPAD methodology against the maximum likelihood
implementation, and to compare within the Bayesian approach the use of
single- vs.~multi-species modelling and weak vs.~strong priors. The
experimental designs presented here follow the ADEMP protocol, which
explictly defines the aims, data-generating mechanisms, estimands,
methods, and performance measures for a given simulation study (Morris
\emph{et al.} 2019).

\hypertarget{experiment-1}{%
\subsubsection{Experiment 1:}\label{experiment-1}}

\textbf{Aim}: To corroborate the results of the Bayesian QPAD model with
the maximum likelihood QPAD model, for single species models.

\textbf{Data-generating mechanisms}: Bird count data are simulated using
a constant singing rate \(\phi\) and constant effective detection radius
\(\tau\). The recording of birds in any survey requires the considering
of a bird's availability and the bird's conditional perceptibility.
Suppose we are sampling at a site that contains \(N\) individuals of
species \(s\). The availability \(p\) is the probability that a bird
gives a cue within a time interval \(t\) of the bird survey, and is
calculated as

\[
  p(t_j) = 1 - \exp\{-t_j \phi\}
\] That is, \(p\) is the proportion of \(N\) individuals that give a cue
and are available to be perceived by an observer.

Then, the perceptibility \(q\) ks the probability that a bird cue is
perceived by an observer, given the bird gives a cue, and is calculated
as

\[
  q(r_k) = \dfrac{\pi\tau^2\left(1-\exp\left[\dfrac{-r_k^2}{\tau^2}\right]\right)}{\pi r_k^2}  
\]

That is, \(q\) is the proportion of \(Np\) individuals that are
perceived by an observer \emph{and} recorded. Therefore, once a survey
is complete, a total of \(n = Npq\) birds are recorded.

Because the QPAD methodology is meant to allow for heterogeneous data
(Sólymos \emph{et al.} 2013), we allow for a random proportion of the
observations of each dataset to be generated using any 4 of the
following sampling protocols:

\begin{itemize}
\tightlist
\item
  P1: Removal Sampling of 0-3 minute, 3-4 minutes, 3-5 minutes, Distance
  Sampling of 0-50m, 50-100m, 100-400m
\item
  P2: Removal Sampling of 0-2 minute, 2-4 minutes, 4-5 minutes, 5-6
  minutes, Distance Sampling of 0-25, 25-50, 50-100, 100-150
\item
  P3: Removal Sampling of 1 minute intervals up to 10 minutes, Distance
  sampling of 25m bins up to 150, then 150 +
\item
  P4: Removal sampling of 0-2 minutes, then 1 minute intervals up to 8
  minutes, Distance sampling of 10m up to 100, then 100m +
\end{itemize}

The proportions of each protocol are drawn from a Dirichlet distribution
with a concentration parameter vector \(\mathbf{a} = \mathbf{1}\).

The \(i\)th sample in a data set represents results from an independent
bird survey. Thus we allow \(N\) to vary by sample by drawing
\(N\sim Poisson(10)\), and adding 1 to each \(N\) to ensure no 0 counts
exist.

For each sampling event \(i\), we first calculate the proportion of
\(N\) that become available in the first time band \(j = 1\), by subbing

\hypertarget{experiment-2}{%
\subsubsection{Experiment 2}\label{experiment-2}}

TO DO

\hypertarget{experiment-3}{%
\subsubsection{Experiment 3}\label{experiment-3}}

TO DO

\clearpage

\hypertarget{results}{%
\section{RESULTS}\label{results}}

TO DO

\clearpage

\hypertarget{discussion}{%
\section{DISCUSSION}\label{discussion}}

TO DO

\clearpage

\hypertarget{conclusions}{%
\section{CONCLUSIONS}\label{conclusions}}

TO DO

\clearpage

\hypertarget{acknowledgements}{%
\section{ACKNOWLEDGEMENTS}\label{acknowledgements}}

TO DO

\hypertarget{author-contributions}{%
\section{AUTHOR CONTRIBUTIONS}\label{author-contributions}}

TO DO

\hypertarget{refs}{}
\leavevmode\hypertarget{ref-morris_using_2019}{}%
Morris, T.P., White, I.R. \& Crowther, M.J. (2019). Using simulation
studies to evaluate statistical methods. \emph{Statistics in Medicine},
\textbf{38}, 2074--2102.

\leavevmode\hypertarget{ref-solymos_calibrating_2013}{}%
Sólymos, P., Matsuoka, S.M., Bayne, E.M., Lele, S.R., Fontaine, P.,
Cumming, S.G., Stralberg, D., Schmiegelow, F.K.A. \& Song, S.J. (2013).
Calibrating indices of avian density from non-standardized survey data:
Making the most of a messy situation (R.B. O'Hara, Ed.). \emph{Methods
Ecol Evol}, \textbf{4}, 1047--1058.

\end{document}
